\documentclass{article}
\usepackage{../../std_header}

\begin{document}
	\title{Computing Nash Equilibria for Strategic Games}
	\author{Haoran Peng}
	\maketitle

Dominance: For two strategies $x_i, x_i'$, we say $x_i$ (weakly) dominates $x_i'$ if for all $x_{-i}$:
\begin{align*}
U_i(x_{-i}; x_i) \ge U_i(x_{-i}; x_i')
\end{align*}
We say $x_i$ strictly dominates $x_i'$ if for all $x_{-i}$:
\begin{align*}
	U_i(x_{-i}; x_i) > U_i(x_{-i}; x_i')
\end{align*}

An even stronger proposition is that $x_i$ dominates $x_i'$ if for all pure counter profiles $\pi_{-1}$:
\begin{align*}
	U_i(\pi_{-i}; x_i) \ge U_i(\pi_{-i}; x_i')
\end{align*}

A strategy  is \textbf{dominant} if it dominates every other strategy. A strategy is \textbf{dominated} if there exists at least one other strategy that dominates it.

A strictly dominant strategy must be played in an NE, because the player should always want to switch to it.

A strictly dominated strategy is clearly bad, and it does not make sense for a rational player to play such strategy. If we assume every play is rational and that each of them knows everyone else is rational: we can eliminate such strictly dominated strategy without eliminate any NE. Note that:
\begin{itemize}
\item NEs might be eliminated if weakly dominated strategies are eliminated
\item A strategy can be dominated by mixed strategies
\end{itemize}

We can find the NEs of the game by iteratively eliminating strictly dominated strategies, and then find the NEs in the residual game.

\newpage
\textbf{Proposition: } In an $n$-player game, strategy profile $x^*$ is an NE iff there exists $w_1,\ldots,w_n \in \mathbb{R}$ (payoffs for every player) such that:
\begin{itemize}
\item $\forall$ player $i$, $\forall \pi_{i,j} \in$ support$(x^*_i)$, $U_i(x^*_{-i}; \pi_{i,j}) = w_i$
\item $\forall$ player $i$, $\forall \pi_{i,j} \notin$ support$(x^*_i)$, $U_i(x^*_{-i}; \pi_{i,j}) \le w_i$
\end{itemize}

In a two-person game, we can enumerate all possible support sets, and solve LPs according to the above constraints. This enumeration is clearly in exponential time. Or use the Lemke-Howson algorithm to find 1 NE, omitted here.
\end{document}















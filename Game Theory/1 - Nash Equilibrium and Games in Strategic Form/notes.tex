\documentclass{article}
\usepackage{../../std_header}

\begin{document}
\title{Nash Equilibrium and Games in Strategic Form}
\author{Haoran Peng}
\maketitle
A Nash Equilibrium (NE) is an $n$-tuple of strategies for the $n$ players such that no player can benefit by \textbf{unilaterally} deviating from its strategy. 

"Unilaterally means" that the player switches its own strategy but all other players do not switch their strategies. "No player can benefit" means that no player can \textbf{strictly} increase its payoff.
\\~\\
\textbf{Nash's Theorem: } Every (finite) game has a mixed Nash equilibrium. 
\\~\\
A finite strategic form game $\Gamma$ with $n$-players consists of:
\begin{itemize}
\item A set $N = \{1,\ldots, n\}$ of players
\item For each player $i\in N$, there is a finite set $S_i = \{1,\ldots, m\}$ of pure strategies.

Let $S = S_1\times S_2\times \cdots \times S_n$ be the set of possible combinations of pure strategies.
\item For each $i\in N$ a payoff function $u_i : S \mapsto \mathbb{R}$, describes the payoff $u_i(s_1, \ldots, s_n)$ to player $i$ under each strategy profile.
\end{itemize}

A mixed strategy $x_i$ for player $i$ is a probability distribution over $S_i$. Let $X_i$ be the set of mixed strategies for player $i$, then $X = X_1 \times \cdots\times X_n$ is the set of all possible profiles of mixed strategies.
\\~\\
Let $x\in X$ be a mixed strategy profile and $s \in S$ be a pure strategy profile. The probability of $s$ under $x$ is:
\begin{align*}
x(s) = \sum_{j=1}^{n} x_j(s_j)
\end{align*}
The expected payoff for player $i$ under a mixed strategy profile $x$ is:
\begin{align*}
U_i(x) = \sum_{s\in S} x(s) \times u_i(s)
\end{align*} 
Then an NE is a profile $x\in X$ such that for every player $i$ and for every $y_i \in X_i$ $\implies$ $U_i(x) \ge U_i(x_{-i}; y_i)$.
\newpage
Claim: A profile $x^*$ is an NE iff for every player $i$ and every pure strategy $\pi_{i,j}$ such that $j\in S_i$, $U_i(x^*)\ge U_i(x^*_{-i}; \pi_{i, j})$.

Proof: If $x^*$ is an NE, it is obvious the inequality holds. In the other direction, first establish that:
\begin{align*}
U_i(x^*_{-i}; x_i) = \sum_{j=1}^{m_i} x_i(j) \times U_i(x^*_{-i}; \pi_{i,j})
\end{align*}
Then we since we know $U_i(x^*)\ge U_i(x^*_{-i}; \pi_{i, j})$:
\begin{align*}
\sum_{j=1}^{m_i} x_i(j) \times U_i(x^*) \ge	U_i(x^*_{-i}; x_i) = \sum_{j=1}^{m_i} x_i(j) \times U_i(x^*_{-i}; \pi_{i,j}) \\
U_i(x^*) \ge	U_i(x^*_{-i}; x_i) = \sum_{j=1}^{m_i} x_i(j) \times U_i(x^*_{-i}; \pi_{i,j})
\end{align*}
\qed

One can go on to prove Nash's Theorem from here, omitted. 

\textbf{Useful Corollary for Nash Equilibria: }

Since $U_i(x^*) = \sum_{j=1}^{m_i} x^*_i(j) \times U_i(x^*_{-i}; \pi_{i,j})$, we can see that $U_i(x^*) =U_i(x^*_{-i}; \pi_{i,j})$ whenever $x^*_i(j) > 0$. If this is not true, there necessarily exist some $U_i(x^*_{-i}; \pi_{i,j}) > U_i(x^*)$, thus a contradiction.
\\~\\
A 2-player game is symmetric if $S_1 = S_2$ (both players have the same strategy sets) and for all $s_1, s_2 \in S_1$, $u_1(s1, s2) = u_2(s2, s1)$. If the payoffs are written in a matrix $A$, then $A^T = A$. Every symmetric game has a symmetric NE, $(x^*_1,x^*_1)$.
\\~\\
A 2-player zero-sum game is a game where for all $s\in S$:
\begin{align*}
u_1(s) + u_2(s) = 0\\
\text{i.e. } u_1(s) = -u_2(s)
\end{align*}
If player 1's payoff matrix is $A_1$ and player 2's is $A_2$, then $A_1 = -A_2$. Given only player 1's payoff matrix, player 1 wants to maximise it and player 2 wants to minimize it.

Suppose player 1's payoff matrix is $A$. If player 1 chooses a mixed strategy $x_1$ (in vector form) and player 2 chooses mixed strategy $x_2$. Then for the profile $x = (x_1, x_2)$, the expected payoff is:
\begin{align*}
U_1(x) = -U_2(x) = x_1^TAx_2 
\end{align*} 
\newpage
\textbf{The Minimax Theorem: } Let a two-player zero-sum game $\Gamma$ be given by a $m_1\times m_2$ matrix $A$ of real numbers. There exists an unique value $v^* \in \mathbb{R}$, such that there exists $x^* = (x^*_1, x^*_2) \in X$ such that:
\begin{enumerate}
\item $((x^*_1)^TA)_j \ge v*$ for $j \in \{1, \cdots, m_2\}$
\item $(Ax^*_2)_j \le v*$ for $j \in \{1, \cdots, m_1\}$
\item Thus $v^* = (x^*_1)^TAx^*_2$
\item The above conditions hold when $x^*$ is any NE. 
\end{enumerate} 
We call any such $x^*$ a minimax profile and $v^*$ the minimax value.
\end{document}














